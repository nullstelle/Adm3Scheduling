\documentclass[11pt,a4paper,notitlepage]{article}
\usepackage{geometry}
\usepackage[utf8]{inputenc}
\usepackage[T1]{fontenc}
\usepackage[english]{babel}
\usepackage{amsmath,amsfonts,amssymb,amsthm}
\usepackage{enumerate}
\usepackage{listings}
\usepackage{algorithmic}
\usepackage{float}
\usepackage[pdftex]{hyperref}
\usepackage{algorithm}
\usepackage{shadethm}

% Define header and footer.
\usepackage{fancyhdr}
\pagestyle{fancy}
\fancyhf{}
\fancyhead[L]{\leftmark}
\fancyhead[R]{\thepage}
\renewcommand{\headrulewidth}{0.5pt}
%\renewcommand{\footrulewidth}{0.5pt}

% Set of numbers.
\providecommand{\R}{\mathbb{R}}
\providecommand{\Z}{\mathbb{Z}}
\providecommand{\N}{\mathbb{N}}
\providecommand{\Q}{\mathbb{Q}}
\providecommand{\C}{\mathbb{C}}

% Complexity zoo.
\renewcommand{\P}{\mathcal{P}}
\providecommand{\NP}{\mathcal{NP}}
\providecommand{\RP}{\mathcal{RP}}
\providecommand{\PSPACE}{\mathcal{PSPACE}}
\providecommand{\coNP}{\text{co-}\NP}
\providecommand{\coRP}{\text{co-}\RP}
\providecommand{\BPP}{\mathcal{BPP}}

% Landau symbols
\providecommand{\Oh}{\mathcal{O}} % big O
\providecommand{\oh}{\mathrm{o}}  % small O



%\DeclareMathOperator{\sgn}{sgn}

\providecommand{\half}{\frac{1}{2}}
\providecommand{\modOp}{\ \mathrm{mod}\ }
\providecommand{\divOp}{\ \mathrm{div}\ }
\providecommand{\norm}[1]{\left\lVert #1 \right\rVert}
\providecommand{\floor}[1]{\left\lfloor #1 \right\rfloor}
\providecommand{\ceiling}[1]{\left\lceil #1 \right\rceil}
\providecommand{\abs}[1]{\left\lvert #1 \right\rvert}

% Environment definitions.
\theoremstyle{plain}
\newshadetheorem{defn}{Definition}[section]
\newtheorem{note}{Notation}[section]
\newtheorem{cor}[defn]{Corollary}
\newtheorem{thm}[defn]{Theorem}
\newtheorem{prop}[defn]{Proposition}
\newtheorem{lemma}[defn]{Lemma}

\theoremstyle{definition}
\newtheorem{remark}[defn]{Remark}
\newtheorem{ex}[defn]{Example}
\newtheorem{exercise}{Exercise}
\setcounter{exercise}{-1}

% Algorithm package.
\renewcommand{\algorithmicrequire}{\textbf{Input:}}
\renewcommand{\algorithmicensure}{\textbf{Output:}}
\renewcommand{\listalgorithmname}{List of Algorithms}
\floatname{algorithm}{Algorithm}

\bibliographystyle{alphadin}

\author{Steffen Suerbier\\Benjamin Labonte}
\title{\sc AdM III: Scheduling and Project Planning}

\begin{document}

% use roman page numbers for the table of contents
\pagenumbering{roman}

\maketitle

\begin{abstract}
Lecture notes taken during the 2012/2013 lecture ``Scheduling and Project Planning'' by Prof. Dr. Rolf H. Möhring.
\end{abstract}

\tableofcontents

% \listofalgorithms
\pagebreak

% Begin sections at 0 (for preface).
% \setcounter{section}{-1}

%\include{preface}

\pagenumbering{arabic}

% Vorlesung 1

\section{Projects and Partial Orders} % (fold)
\label{sec:projects_and_partial_orders}

What makes a project?
\begin{itemize}
  \item activities or jobs $j\in V = \{1,2,\cdots,n\}$
  \item job data
  \begin{itemize}
    \item processing time/duration (deterministic/random)
    \item resource consumption
    \item processing cost
  \end{itemize}
  \item project data
  \begin{itemize}
    \item available resources
    \item limited budget
  \end{itemize}
  \item project rules for carrying out the project
  \begin{itemize}
    \item temporal conditions
    \item resource conditions
    \item setups
  \end{itemize}
\end{itemize}

Precedence constraints, i.e. job $j$ has to wait for $i$ define a partial ordering on $V$. We write $i<j$ if $j$ must wait for $i$.

\begin{defn}[Partial Ordering]
  A (strict) partial order is a binary relation $<$ over a set $V$ which is
  \begin{enumerate}[(i)]
    \item assymmetric: $i<j \Rightarrow j \not < i$
    \item transitive: $i<j,\ j<k \Rightarrow i < k$
  \end{enumerate}
\end{defn}

\begin{defn}[Transitive Reduction of $<$, Transitive closure]
  The transitive reduction of the finite partial order $(V,<)$ is the digraph $G = (V,E)$ with the edges
  \[ E = \{(i,j)\ |\ i,j\in V,\ i<j,\ \nexists k: i < k < j\}. \]
  In the context of scheduling this is also called activity-on-node diagram. For any graph $G = (V,E)$ the edge set defined by
  \[ E^{trans} = \{(i,j)\ |\ \exists \text{finite sequence } i=i_{0},\cdots,i_{k} = j: (i_{j}, i_{j+1})\in E\}\]
  denotes the transitive closure of $E$.
\end{defn}

\begin{ex}[Bridge Construction Project]
  % TODO(steffen): add image
\end{ex}

\begin{defn}
  \begin{align*}
    Pred_{G}(i) &= \{j\in V \ |\ j < i\}\quad &&\text{set of predecessors}\\
    Suc_{G}(i) &= \{j\in V \ |\ j > i\}\quad &&\text{set of successors}\\
    ImPred_{G}(i) &= \{j\in V \ |\ (j,i)\in E\}\quad &&\text{set of predecessors}\\
    ImSuc_{G}(i) &= \{j\in V \ |\ (i,j)\in E\}\quad &&\text{set of predecessors}
  \end{align*}
  We call an element maximal if it has no successors, minimal if it has no predecessors. If there is only one maximal (minimal) element it is called greatest (smallest). $i,j$ are comparable ($i \sim_{G} j$) if $i<j$ or $j>i$ otherwise they are incomparable ($i\parallel_{G} j$).
\end{defn}

\begin{lemma}
  \label{lemma:gen_rel}
  Let $(V,<)$ be a finite partial order and $(V,E)$ be its transitive reduction.
  Then $E$ is the smallest (under $\leq$) binary relation, s.t. $E^{trans} =\ <$.
\end{lemma}

\begin{proof}
  Show that $E^{trans} =\ <$. Let $i<j$ and $(i,j)\not\in E$. Then there is a $k\in V$ s.t. $i<k<j$. As $<$ is acyclic and finite, by iterating, we obtain a finite sequence $i=i_{0}\rightarrow \cdots \rightarrow j$.
\end{proof}

\begin{exercise}
  Show that $E^{trans}$ exists.
\end{exercise}

\begin{remark}
  Lemma \ref{lemma:gen_rel} does not hold for relations with cycles or infinite ground sets.
\end{remark}

\begin{exercise}
  Formulate an algorithm for constructing the transitive closure of a digraph.
\end{exercise}

\begin{exercise}
  Formulate an algorithm for constructing the transitive reduction of a partial order or a directed acyclic graph.
\end{exercise}

\begin{exercise}
  Prove an $\Oh(n^{3})$ upper bound on the running times of the algorithms.
\end{exercise}

% section projects_and_partial_orders (end)

%\appendix
%\include{appendix}

% bind in bibliography
%\bibliography{adm3}

\end{document}
