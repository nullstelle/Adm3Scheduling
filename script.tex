\documentclass[11pt,a4paper,notitlepage]{article}
\usepackage{geometry}
\usepackage[utf8]{inputenc}
\usepackage[T1]{fontenc}
\usepackage[english]{babel}
\usepackage{amsmath,amsfonts,amssymb,amsthm}
\usepackage{enumerate}
\usepackage{listings}
\usepackage{algorithmic}
\usepackage{float}
\usepackage[pdftex]{hyperref}
\usepackage{algorithm}
\usepackage{shadethm}
\usepackage{dsfont}

% Define header and footer.
\usepackage{fancyhdr}
\pagestyle{fancy}
\fancyhf{}
\fancyhead[L]{\leftmark}
\fancyhead[R]{\thepage}
\renewcommand{\headrulewidth}{0.5pt}
%\renewcommand{\footrulewidth}{0.5pt}

% Set of numbers.
\providecommand{\R}{\mathbb{R}}
\providecommand{\Z}{\mathbb{Z}}
\providecommand{\N}{\mathbb{N}}
\providecommand{\Q}{\mathbb{Q}}
\providecommand{\C}{\mathbb{C}}

% Complexity zoo.
\renewcommand{\P}{\mathcal{P}}
\providecommand{\NP}{\mathcal{NP}}
\providecommand{\RP}{\mathcal{RP}}
\providecommand{\PSPACE}{\mathcal{PSPACE}}
\providecommand{\coNP}{\text{co-}\NP}
\providecommand{\coRP}{\text{co-}\RP}
\providecommand{\BPP}{\mathcal{BPP}}

% Landau symbols
\providecommand{\Oh}{\mathcal{O}} % big O
\providecommand{\oh}{\mathrm{o}}  % small O



%\DeclareMathOperator{\sgn}{sgn}

\providecommand{\half}{\frac{1}{2}}
\providecommand{\modOp}{\ \mathrm{mod}\ }
\providecommand{\divOp}{\ \mathrm{div}\ }
\providecommand{\norm}[1]{\left\lVert #1 \right\rVert}
\providecommand{\floor}[1]{\left\lfloor #1 \right\rfloor}
\providecommand{\ceiling}[1]{\left\lceil #1 \right\rceil}
\providecommand{\abs}[1]{\left\lvert #1 \right\rvert}

% Environment definitions.
\theoremstyle{plain}
\newshadetheorem{defn}{Definition}[section]
\newshadetheorem{problem}{Problem}
\newtheorem{note}{Notation}[section]
\newtheorem{cor}[defn]{Corollary}
\newshadetheorem{theorem}[defn]{Theorem}
\newtheorem{prop}[defn]{Proposition}
\newtheorem{lemma}[defn]{Lemma}

\theoremstyle{definition}
\newtheorem{remark}[defn]{Remark}
\newtheorem{example}[defn]{Example}
\newtheorem{exercise}{Exercise}
\setcounter{exercise}{-1}
\newtheorem*{claim}{Claim}

% Algorithm package.
\renewcommand{\algorithmicrequire}{\textbf{Input:}}
\renewcommand{\algorithmicensure}{\textbf{Output:}}
\renewcommand{\listalgorithmname}{List of Algorithms}
\floatname{algorithm}{Algorithm}

\bibliographystyle{alphadin}

\author{Steffen Suerbier\\Benjamin Labonte}
\title{\sc AdM III: Scheduling and Project Planning}

\begin{document}

% use roman page numbers for the table of contents
\pagenumbering{roman}

\maketitle

\begin{abstract}
Lecture notes taken during the 2012/2013 lecture ``Scheduling and Project
Planning'' by Prof. Dr. Rolf H. Möhring.
\end{abstract}

\tableofcontents

% \listofalgorithms
\pagebreak

% Begin sections at 0 (for preface).
% \setcounter{section}{-1}

%\include{preface}

\pagenumbering{arabic}

% Lecture 1 (2012-10-18)

\section{Projects and Partial Orders}
\label{sec:projects_and_partial_orders}

What makes a project?
\begin{itemize}
  \item activities or jobs $j\in V = \{1,2,\cdots,n\}$
  \item job data
  \begin{itemize}
    \item processing time/duration (deterministic/random)
    \item resource consumption
    \item processing cost
  \end{itemize}
  \item project data
  \begin{itemize}
    \item available resources
    \item limited budget
  \end{itemize}
  \item project rules for carrying out the project
  \begin{itemize}
    \item temporal conditions
    \item resource conditions
    \item setups
  \end{itemize}
\end{itemize}

Precedence constraints, i.e. job $j$ has to wait for $i$ define a partial
ordering on $V$. We write $i<j$ if $j$ must wait for $i$.

\begin{defn}[Partial Ordering]
  A (strict) partial order is a binary relation $<$ over a set $V$ which is
  \begin{enumerate}[(i)]
    \item assymmetric: $i<j \Rightarrow j \not < i$
    \item transitive: $i<j,\ j<k \Rightarrow i < k$
  \end{enumerate}
\end{defn}

\begin{defn}[Transitive Reduction of $<$, Transitive closure]
  The transitive reduction of the finite partial order $(V,<)$ is the digraph
  $G = (V,E)$ with the edges
  \[ E = \{(i,j)\ |\ i,j\in V,\ i<j,\ \nexists k: i < k < j\}. \]
  In the context of scheduling this is also called activity-on-node diagram.
  For any graph $G = (V,E)$ the edge set defined by
  \[ E^{trans} = \{(i,j)\ |\ \exists \text{finite sequence } i=i_{0},\cdots,i_{k} = j: (i_{j}, i_{j+1})\in E\}\]
  denotes the transitive closure of $E$.
\end{defn}

\begin{example}[Bridge Construction Project]
  % TODO(steffen): add image
\end{example}

\begin{defn}
  \begin{align*}
    Pred_{G}(i) &= \{j\in V \ |\ j < i\}\quad &&\text{set of predecessors}\\
    Suc_{G}(i) &= \{j\in V \ |\ j > i\}\quad &&\text{set of successors}\\
    ImPred_{G}(i) &= \{j\in V \ |\ (j,i)\in E\}\quad &&\text{set of predecessors}\\
    ImSuc_{G}(i) &= \{j\in V \ |\ (i,j)\in E\}\quad &&\text{set of predecessors}
  \end{align*}
  We call an element maximal if it has no successors, minimal if it has no
  predecessors. If there is only one maximal (minimal) element it is called
  greatest (smallest). $i,j$ are comparable ($i \sim_{G} j$) if $i<j$ or $j>i$
  otherwise they are incomparable ($i\parallel_{G} j$).
\end{defn}

\begin{lemma}
  \label{lemma:gen_rel}
  Let $(V,<)$ be a finite partial order and $(V,E)$ its transitive reduction.
  Then $E$ is the smallest (under $\subset$) binary relation, s.t.
  $E^{trans} =\ <$.
\end{lemma}

\begin{proof}
  Show that $E^{trans} =\ <$. Let $i<j$ and $(i,j)\not\in E$. Then there is a
  $k\in V$ s.t. $i<k<j$. As $<$ is acyclic and finite, by iterating, we obtain
  a finite sequence $i=i_{0}\rightarrow \cdots \rightarrow j$.
\end{proof}

\begin{remark}
  Lemma \ref{lemma:gen_rel} does not hold for relations with cycles or infinite
  ground sets.
\end{remark}

\begin{exercise}
  Show that $E^{trans}$ exists.
\end{exercise}

\begin{exercise}
  Formulate an algorithm for constructing the transitive closure of a digraph.
\end{exercise}

\begin{exercise}
  Formulate an algorithm for constructing the transitive reduction of a partial
  order or a directed acyclic graph.
\end{exercise}

\begin{exercise}
  Prove an $\Oh(n^{3})$ upper bound on the running times of the algorithms.
\end{exercise}

% section projects_and_partial_orders (end)

\section{Scheduling subject to scarce resources: Introduction and complexity}
\label{sec:resource_constrained_scheduling}

% Lecture 4 (2012-10-26)

\begin{example}
  Consider the following standard model in parallel processing environments.
  Every long job is cut into pieces of length $1$. $m$-machine problems for
  $m\geq 2$, $G$ precedence constraints, $x_j = 1, \kappa = C_{max}$.

  \paragraph{Results}

  $m=2$ polynomially solvable\\
  $m=3$ open (for a long time)\\
  $m \geq 4$ $\NP-complete$

  \paragraph{Idea for $m=2$}
  Match jobs that are scheduled in same time slot. Use incomparability graph
  $Incomp(G)$ with vertices $V=\text{set of jobs}$ and edges
  $(i,j)\in E\Leftrightarrow i\parallel_G j$.
  % TODO(steffen): concrete example
\end{example}

\begin{theorem}
  $C_{max}^{OPT, G}(\mathds{1}) = \text{maximum size of a matching in } Incomp(G) + \#\text{unmatched jobs}$ [Fujii et al. 67,71].
\end{theorem}

\begin{proof}
  Construct a schedule from the scheduling from the matching requires
  "uncrossing" of crossings.

  crossing:
  % TODO(steffen): add image for crossings
  Assume $(a,c)$ and $(b,d)$ are precedence constraints, then a matching of $a$
  with $d$ and $b$ with $c$ would lead to an infeasible schedule. The
  uncrossing is simply the matching $a$ with $b$ and $c$ with $d$.

  \begin{claim}
    $\forall$ matchings $M$ there is a non-crossing matching of the same size
    that can be obtained by uncrossing crossings in any order
  \end{claim}

  \begin{proof}
    Every oncrossing reduces the number of crossings. Suppose not, i.e.
    consider $(u_{1}, u_{2}), (u_{3}, u_{4})$, uncross the matching
    $u_{1}\leftrightarrow u_{4}, u_{2}\leftrightarrow u_{3}$ and this gives a
    new crossing $u_{1}\leftrightarrow u_{2}, v_{1}\leftrightarrow v_{2}$
    $\Rightarrow \exists u_{1}<v_{1}, v_{2}<u_{2}: (v_{1},v_{2})\in M$.
    % TODO(steffen): add pictures for this
    But that means that before uncrossing $(v_{1},v_{2})$ crossed with
    $(u_{1}m u_{4})$ which no longer exists.

    This shows that for every new crossing an old crossing involving the same
    edges is uncrossed.
  \end{proof}

  Now assume that $M$ is non-crossing. Turn that into a schedule.
  \begin{enumerate}
    % TODO(steffen): picture missing
    \item order the matched edges [picture], if one of $\{a,b\}$ is before on
      of $\{c,d\}$ in $G^{trans}$ (possible because of non-crossing)
    \item order unmatched jobs and matched edges [picture] if $a$ is before
      one of $\{c,d\}$ in $G^{trans}$. (similar: [picture matching before job]).
      This is well defined since $c\parallel_{G} d$.
    \item take any linear extension of the transitive closure of $\prec$
      defined in (1), (2) and $a<_{G}$ between unmatched jobs.
    \item Replace matching edges by their two end points. This defines $G^{*}$
      on $V$.
    \item Consider $ES_{G^{*}}(\mathds{1})$
  \end{enumerate}
\end{proof}

\begin{example}[continued]
  % TODO(steffen): add picture
  TODO
\end{example}

\begin{problem}[$n$-\textsc{Machine Problem}]
  % TODO(steffen): find a better way to define problems
  \quad
  \begin{algorithmic}
    \REQUIRE $G,m,t,x=\mathds{1}$
    \ENSURE Is there a feasible schedule of length $\leq t$?
  \end{algorithmic}
\end{problem}

\begin{theorem}
  The $m$-machine problem is $\NP-complete$ for $m\geq 4$.
\end{theorem}

\begin{proof}
  Reduction from \textsc{Clique}. Vertex jobs $u_{1},\cdots,u_{n}$, edge jobs
  $u_{ij}$, precedence constraints $u_{ij}>u_{i}, u_{ij}>u_{j}$ and dummy jobs
  which provide a frame. $k=3$ and $t=3$ (always).

  \paragraph{Idea} Can schedule $k$ vertex jobs in slot $A$, $k\choose 2$ edge
  jobs and remaining vertex jobs in slot $B$ and remaining edge jobs in slot
  $C$.

  % TODO(steffen): picture and text, up to this point this is totally useless

  If $G$ constains a clique of size $k$ then we can schedule the vertex jobs
  from the clique in the first time slot, the edge jobs of the clique and the
  remaining vertex jobs in the second slot and the remaining edge jobs in the
  third time slot. So we get a schedule in $3$ time slots.

  If $G$ does not contain a cliqe of size $k$ then one machine is idle in time
  slot 2 and we would need 4 time slots.
\end{proof}

\begin{exercise}
  Show that Jacksons's rule constructs an optimal schedule for $L_{max}$ on one
  machine.
\end{exercise}

\begin{exercise}
  Show that the 3-machine problem can be solved in polynomial time for
  precedence constrains of fixed with (i.e. the size of the largest antichain
  is bounded by a constant).
\end{exercise}

% section scheduling_subject_to_scarce_resources_introduction_and_complexity (end)

%\appendix
%\include{appendix}

% bind in bibliography
%\bibliography{adm3}

\end{document}
